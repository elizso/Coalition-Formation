\begin{abstract}
Agents in economics and politics often form coalitions to compete in games and markets.
Yet membership is fluid: any participant may perpetrate, meaning leave a coalition, whenever doing so promises greater benefit.
Because one perpetration can trigger a cascade of further perpetrations,
agents must consider not only the immediate gain but also the entire chain of possible consequences,
a perspective known as \emph{farsightedness}.
Assuming symmetric agents who act as farsighted perpetrators and adopt a maximin strategy,
we present a dynamic-programming algorithm that computes stable coalition-structure equilibria in such coalition-formation games.
We apply the algorithm to study Cournot competition among coalitions, showing how farsightedness influences the resulting equilibrium and might yield surprising outcomes.
\end{abstract}

%%% Use this command to specify a few keywords describing your work.
%%% Keywords should be separated by commas.

%\keywords{Algorithm, Coalition Formation, Dynamic Programming, Blocking}
