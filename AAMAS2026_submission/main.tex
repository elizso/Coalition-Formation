%%%%%%%%%%%%%%%%%%%%%%%%%%%%%%%%%%%%%%%%%%%%%%%%%%%%%%%%%%%%%%%%%%%%%%%%

%%% LaTeX Template for AAMAS-2026 (based on sample-sigconf.tex)
%%% Prepared by the AAMAS-2026 Publication Chairs based on the version from AAMAS-2025. 

%%%%%%%%%%%%%%%%%%%%%%%%%%%%%%%%%%%%%%%%%%%%%%%%%%%%%%%%%%%%%%%%%%%%%%%%

%%% Start your document with the \documentclass command.


%%% == IMPORTANT ==
%%% Use the first variant below for the final paper (including author information).
%%% Use the second variant below to anonymize your submission (no author information shown).
%%% For further information on anonymity and double-blind reviewing, 
%%% please consult the call for paper information
%%% https://cyprusconferences.org/aamas2026/submission-instructions/

%%%% For anonymized submission, use this
\documentclass[sigconf,anonymous]{aamas} 

%%%% For camera-ready, use this
%\documentclass[sigconf]{aamas} 


%%% Load required packages here (note that many are included already).

\usepackage{balance} % for balancing columns on the final page
\usepackage{algorithm}
\usepackage[noend]{algpseudocode}

\newcommand{\ykc}[1]{{\color{blue} #1}}

%%%%%%%%%%%%%%%%%%%%%%%%%%%%%%%%%%%%%%%%%%%%%%%%%%%%%%%%%%%%%%%%%%%%%%%%

%%% AAMAS-2026 copyright block (do not change!)

\setcopyright{ifaamas}
\acmConference[AAMAS '26]{Proc.\@ of the 25th International Conference
on Autonomous Agents and Multiagent Systems (AAMAS 2026)}{May 25 -- 29, 2026}
{Paphos, Cyprus}{C.~Amato, L.~Dennis, V.~Mascardi, J.~Thangarajah (eds.)}
\copyrightyear{2026}
\acmYear{2026}
\acmDOI{}
\acmPrice{}
\acmISBN{}


%%%%%%%%%%%%%%%%%%%%%%%%%%%%%%%%%%%%%%%%%%%%%%%%%%%%%%%%%%%%%%%%%%%%%%%%

%%% == IMPORTANT ==
%%% Use this command to specify your submission number.
%%% In anonymous mode, it will be printed on the first page.

\acmSubmissionID{<<submission id>>}

%%% Use this command to specify the title of your paper.

\title[Stable Coalition Structures]{Stable Coalition Structures of Symmetric\\
Coalition Formation Games with Farsighted Perpetrators}

%%% Provide names, affiliations, and email addresses for all authors.

\author{Eliz B.~So}
\affiliation{
  \institution{Australian National University}
  \city{Canberra}
  \country{Australia }}
\email{eliz.so@anu.edu.au}

\author{Yun Kuen Cheung}
\affiliation{
  \institution{Australian National University}
  \city{Canberra}
  \country{Australia}}
\email{yunkuen.cheung@anu.edu.au}

%%% Use this environment to specify a short abstract for your paper.

\begin{abstract}
To be written.
\end{abstract}

%%% Use this command to specify a few keywords describing your work.
%%% Keywords should be separated by commas.

\keywords{Algorithms, Coalition Formation}


%%%%%%%%%%%%%%%%%%%%%%%%%%%%%%%%%%%%%%%%%%%%%%%%%%%%%%%%%%%%%%%%%%%%%%%%

%%% Include any author-defined commands here.
         
\newcommand{\BibTeX}{\rm B\kern-.05em{\sc i\kern-.025em b}\kern-.08em\TeX}

%%%%%%%%%%%%%%%%%%%%%%%%%%%%%%%%%%%%%%%%%%%%%%%%%%%%%%%%%%%%%%%%%%%%%%%%

\begin{document}

%%% The following commands remove the headers in your paper. For final 
%%% papers, these will be inserted during the pagination process.

\pagestyle{fancy}
\fancyhead{}

%%% The next command prints the information defined in the preamble.

\maketitle 

%%%%%%%%%%%%%%%%%%%%%%%%%%%%%%%%%%%%%%%%%%%%%%%%%%%%%%%%%%%%%%%%%%%%%%%%

\section{Introduction}

\ykc{Read [Ray, Vohra 1997 in JET] and [Ray, Vohra 1999 in GEB] to see how they motivate the problem.}

\noindent \ykc{Distinguish between characteristic function and partition function. RV99 introduction has some relevant discussion.}

\noindent \ykc{RV99 abstract wrote ``we provide a uniqueness theorem and apply our results to a Cournot oligopoly''. We should understand what the theorem is and how their results are relevant to ours.}

\subsection*{Related Work}
Our model differs from other standard cooperative game models. Standard cooperative games using Shapley or core values or even Hedonic games use characteristic function which is independent of outsiders. Their coalition payoffs are fixed regardless of what others do, unlike our model, the payoff is partition dependent. The payoff in our model would have to be calculated recursively by simulating what the best response would be. 

In addition, standard cooperative games allow solution concepts such as Shapley or core values to be applied directly to the characteristic function. However, in our model, these concepts would only be applied after computing the partition-dependent payoffs. 

Moreover, Hedonic games requires a different input to our model. Each agent in Hedonic games has a preference ordering coalitions that contain them, where the preference is the factor of whether an agent stays in the coalition. Yet, agents in our model decide whether to stay in one coalition by their own payoff which is dependent to other agents' movements. % introduction and related work
% create more .tex files as needed, say one .tex file per section; this can avoid some trouble when both of us are making edits before paper deadline

%%%%%%%%%%%%%%%%%%%%%%%%%%%%%%%%%%%%%%%%%%%%%%%%%%%%%%%%%%%%%%%%%%%%%%%

%\section{Citations and References}
%  
%The use of the \BibTeX\ to prepare your list of references is highly 
%recommended. To include the references at the end of your document, put 
%the following two commands just before the `\verb|\end{document}|' 
%command in your source file:
%%
%\begin{verbatim}
%   \bibliographystyle{ACM-Reference-Format}
%   \bibliography{sample}
%\end{verbatim}
%%
%Here we assume that `\texttt{sample.bib}' is the name of your 
%\BibTeX\ file. Use the `\verb|\cite|' command to produce citations 
%to your references. Here are a few examples for citations of journal 
%articles~\cite{GrKr96,WoJe95}, books~\cite{Knu97}, articles in 
%conference proceedings~\cite{Hag1993}, technical reports~\cite{Har78},
%Master's and PhD theses~\cite{Ani03,Cla85}, online videos~\cite{Oba08}, 
%datasets~\cite{AnMC13}, and patents~\cite{Sci09}. Both citations and 
%references are numbered by default. 
%
%Make sure you provide complete and correct bibliographic information 
%for all your references, and list authors with their full names 
%(``Donald E.\ Knuth'') rather than just initials (``D.\ E.\ Knuth''). 

%%%%%%%%%%%%%%%%%%%%%%%%%%%%%%%%%%%%%%%%%%%%%%%%%%%%%%%%%%%%%%%%%%%%%%%%

%%% The acknowledgments section is defined using the "acks" environment
%%% (rather than an unnumbered section). The use of this environment 
%%% ensures the proper identification of the section in the article 
%%% metadata as well as the consistent spelling of the heading.

\begin{acks}
If you wish to include any acknowledgments in your paper (e.g., to 
people or funding agencies), please do so using the `\texttt{acks}' 
environment. Note that the text of your acknowledgments will be omitted
if you compile your document with the `\texttt{anonymous}' option.
\end{acks}

%%%%%%%%%%%%%%%%%%%%%%%%%%%%%%%%%%%%%%%%%%%%%%%%%%%%%%%%%%%%%%%%%%%%%%%%

%%% The next two lines define, first, the bibliography style to be 
%%% applied, and, second, the bibliography file to be used.

\bibliographystyle{ACM-Reference-Format} 
\bibliography{coalition}

%%%%%%%%%%%%%%%%%%%%%%%%%%%%%%%%%%%%%%%%%%%%%%%%%%%%%%%%%%%%%%%%%%%%%%%%

\end{document}

%%%%%%%%%%%%%%%%%%%%%%%%%%%%%%%%%%%%%%%%%%%%%%%%%%%%%%%%%%%%%%%%%%%%%%%%

