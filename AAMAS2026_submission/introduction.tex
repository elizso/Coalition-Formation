\section{Introduction}

\ykc{Read [Ray, Vohra 1997 in JET] and [Ray, Vohra 1999 in GEB] to see how they motivate the problem.}

\noindent \ykc{Distinguish between characteristic function and partition function. RV99 introduction has some relevant discussion.}

\noindent \ykc{RV99 abstract wrote ``we provide a uniqueness theorem and apply our results to a Cournot oligopoly''. We should understand what the theorem is and how their results are relevant to ours.}

\subsection*{Related Work}
Our model differs from other standard cooperative game models. Standard cooperative games using Shapley or core values or even Hedonic games use characteristic function which is independent of outsiders. Their coalition payoffs are fixed regardless of what others do, unlike our model, the payoff is partition dependent. The payoff in our model would have to be calculated recursively by simulating what the best response would be. 

In addition, standard cooperative games allow solution concepts such as Shapley or core values to be applied directly to the characteristic function. However, in our model, these concepts would only be applied after computing the partition-dependent payoffs. 

Moreover, Hedonic games requires a different input to our model. Each agent in Hedonic games has a preference ordering coalitions that contain them, where the preference is the factor of whether an agent stays in the coalition. Yet, agents in our model decide whether to stay in one coalition by their own payoff which is dependent to other agents' movements.