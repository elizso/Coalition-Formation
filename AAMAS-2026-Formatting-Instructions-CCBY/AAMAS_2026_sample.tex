%%%%%%%%%%%%%%%%%%%%%%%%%%%%%%%%%%%%%%%%%%%%%%%%%%%%%%%%%%%%%%%%%%%%%%%%

%%% LaTeX Template for AAMAS-2026 (based on sample-sigconf.tex)
%%% Prepared by the AAMAS-2026 Publication Chairs based on the version from AAMAS-2025. 

%%%%%%%%%%%%%%%%%%%%%%%%%%%%%%%%%%%%%%%%%%%%%%%%%%%%%%%%%%%%%%%%%%%%%%%%

%%% Start your document with the \documentclass command.


%%% == IMPORTANT ==
%%% Use the first variant below for the final paper (including author information).
%%% Use the second variant below to anonymize your submission (no author information shown).
%%% For further information on anonymity and double-blind reviewing, 
%%% please consult the call for paper information
%%% https://cyprusconferences.org/aamas2026/submission-instructions/

%%%% For anonymized submission, use this
\documentclass[sigconf]{aamas} 

%%%% For camera-ready, use this
%\documentclass[sigconf]{aamas} 


%%% Load required packages here (note that many are included already).

\usepackage{balance} % for balancing columns on the final page

%%%%%%%%%%%%%%%%%%%%%%%%%%%%%%%%%%%%%%%%%%%%%%%%%%%%%%%%%%%%%%%%%%%%%%%%

%%% AAMAS-2026 copyright block (do not change!)

\setcopyright{ifaamas}
\acmConference[AAMAS '26]{Proc.\@ of the 25th International Conference
on Autonomous Agents and Multiagent Systems (AAMAS 2026)}{May 25 -- 29, 2026}
{Paphos, Cyprus}{C.~Amato, L.~Dennis, V.~Mascardi, J.~Thangarajah (eds.)}
\copyrightyear{2026}
\acmYear{2026}
\acmDOI{}
\acmPrice{}
\acmISBN{}


%%%%%%%%%%%%%%%%%%%%%%%%%%%%%%%%%%%%%%%%%%%%%%%%%%%%%%%%%%%%%%%%%%%%%%%%

%%% == IMPORTANT ==
%%% Use this command to specify your submission number.
%%% In anonymous mode, it will be printed on the first page.

\acmSubmissionID{<<submission id>>}

%%% Use this command to specify the title of your paper.

\title[AAMAS-2026 Formatting Instructions]{Formatting Instructions for the 25th International Conference on Autonomous Agents and Multiagent Systems}

%%% Provide names, affiliations, and email addresses for all authors.

\author{Eliz B. So}
\affiliation{
  \institution{Australian National University}
  \city{Canberra}
  \country{Australia }}
\email{eliz.so@anu.edu.au}

\author{Yun Keun Cheung (Marco)}
\affiliation{
  \institution{Australian National University }
  \city{Canberra}
  \country{Australia}}
\email{yunkeun.cheung@anu.edu.au}

%%% Use this environment to specify a short abstract for your paper.

\begin{abstract}
This document outlines the formatting instructions for submissions to
AAMAS-2026. You can use its source file as a template when writing 
your own paper. It is based on the file `\texttt{sample-sigconf.tex}'
distributed with the ACM article template for \LaTeX\@.
\end{abstract}

%%% Use this command to specify a few keywords describing your work.
%%% Keywords should be separated by commas.

\keywords{Algorithms, Coalition Formation}

%%%%%%%%%%%%%%%%%%%%%%%%%%%%%%%%%%%%%%%%%%%%%%%%%%%%%%%%%%%%%%%%%%%%%%%%

%%% Include any author-defined commands here.
         
\newcommand{\BibTeX}{\rm B\kern-.05em{\sc i\kern-.025em b}\kern-.08em\TeX}

%%%%%%%%%%%%%%%%%%%%%%%%%%%%%%%%%%%%%%%%%%%%%%%%%%%%%%%%%%%%%%%%%%%%%%%%

\begin{document}

%%% The following commands remove the headers in your paper. For final 
%%% papers, these will be inserted during the pagination process.

\pagestyle{fancy}
\fancyhead{}

%%% The next command prints the information defined in the preamble.

\maketitle 

%%%%%%%%%%%%%%%%%%%%%%%%%%%%%%%%%%%%%%%%%%%%%%%%%%%%%%%%%%%%%%%%%%%%%%%%

\section{Introduction}

This document explains the main features of the `\texttt{aamas}' 
document class, which is essentially identical to the `\texttt{acmart}'
document class provided by the ACM. The only difference is a minor 
modification to allow for the correct copyright attribution to IFAAMAS.
For detailed documentation of the original document class, please refer
to the relevant website maintained by the~ACM:
%
\begin{center}
\url{https://www.acm.org/publications/proceedings-template}
\end{center}
%
The first command in your source file should be either one of these:
\begin{verbatim}
    \documentclass[sigconf,anonymous]{aamas}
    \documentclass[sigconf]{aamas}
\end{verbatim}
%
The first variant should be
used when you submit your paper for blind review; it will replace the names of the authors with the submission number.
The second variant should be used for final papers. 

Make sure your paper includes the correct copyright information and 
the correct specification of the \emph{ACM Reference Format}. Both of 
these will be generated automatically if you include the correct 
\emph{copyright block} as shown in the source file of this document.

Modifying the template---e.g., by changing margins, typeface sizes, 
line spacing, paragraph or list definitions---or making excessive use 
of the `\verb|\vspace|' command to manually adjust the vertical spacing 
between elements of your work is not allowed. You risk getting your 
submission rejected (or your final paper excluded from the proceedings) 
in case such modifications are discovered. The `\texttt{aamas}' document 
class requires the use of the \textit{Libertine} typeface family, which 
should be included with your \LaTeX\ installation. Please do not use 
other typefaces instead.

Please consult the \emph{Call for Papers} for information on matters 
such as the page limit or anonymity requirements. It is available from
the conference website:
%
\begin{center}
\url{https://cyprusconferences.org/aamas2026/}
\end{center}
%
To balance the columns on the final page of your paper, use the 
`\texttt{balance}' package and issue the `\verb|\balance|' command
 somewhere in the text of what would be the first column of the last 
 page without balanced columns. This will be required for final papers.

%%%%%%%%%%%%%%%%%%%%%%%%%%%%%%%%%%%%%%%%%%%%%%%%%%%%%%%%%%%%%%%%%%%%%%%%

\section{Literature Review}

You will be assigned a submission number when you register the abstract 
of your paper on \textit{OpenReview}. Include this number in your 
document using the `\verb|\acmSubmissionID|' command.

Then use the familiar commands to specify the title and authors of your
paper in the preamble of the document. The title should be appropriately 
capitalised (meaning that every `important' word in the title should 
start with a capital letter). For the final version of your paper, make 
sure to specify the affiliation and email address of each author using 
the appropriate commands. Specify an affiliation and email address 
separately for each author, even if two authors share the same 
affiliation. You can specify more than one affiliation for an author by 
using a separate `\verb|\affiliation|' command for each affiliation.

Provide a short abstract using the `\texttt{abstract}' environment.
 
Finally, specify a small number of keywords characterising your work, 
using the `\verb|\keywords|' command. 

%%%%%%%%%%%%%%%%%%%%%%%%%%%%%%%%%%%%%%%%%%%%%%%%%%%%%%%%%%%%%%%%%%%%%%%%

\section{Notations}


\subsection{Coalition Structure (CS)}

A coalition structure (CS) is the partition of the players into one or more coalitions. For example, when $n=8$, some examples of CS are $\{\{1, 2, 3\}, \{4, 5\}, \{6, 7, 8\}\}$, $\{\{1, 7\}, \{2\}, \{3, 5, 8\}, \{4, 6\}\}$, and $\{\{1, 2, 3, 4, 5, 6, 7, 8\}\}$. The last CS where all players form one single coalition is called a grand coalition. 

In the symmetric game, all players are treated equally. As a result, two CSs that are equivalent up to a relabeling of players yield the same reward distribution. Also, within a given CS, all players belonging to coalitions of the same size receive the same rewards. For example, consider the two CSs $\{1, 2, 3\}, \{4, 5, 6\}, \{7, 8\}$ and $\{1, 3, 7\}, \{2, 5, 8\}, \{4, 6\}$ which are equivalent up to a relabeling of players. The rewards received by players 1, 2, 3, 4, 5, 6 in the first CS are the same as those received by players
1, 3, 7, 2, 5, 8 in the second CS.

This model differs from the standard cooperative game model underpinning Shapley and core values. % TODO% 
%%%%%%%%%%%%%%%%%%%%%%%%%%%%%%%%%%%%%%%%%%%%%%%%%%%%%%%%%%%%%%%%%%%%%%%%

\section{Dynamic Betrayal Model}

In general, coalition formation dynamics can involve both splitting and merging of coalitions, making the analysis complex. We focus on a simpler model in which only splitting is allowed. Initially, all players belong to a single grand coalition, which may undergo a sequence of splits. This process continues until no player has an incentive to leave her current coalition and form a smaller one; we will define incentive formally below. In each coalition of size $i \geq 2$, there is a player who may initiate a betrayal by pulling at most $i$ players (including herself) away from the rest of the coalition.


\subsection{Nash Equilibrium}

The definition of Nash Equilbrium is the set of strategies such that no player can unilaterally deviate from their strategy to get a higher payoff, given the strategies of all other players. Formally, we can let $A$ to be the set of players, $S_i$ to be the set of strategies available to players, $S$ as the strategy profiles, $u_i$ as the payoff function of each player $i$. Then $s^*\in S$ is a Nash Equilibrium of a strategic gaame if and only if $\forall i \in A$ and $\forall s_i \in S_i$, 

\begin{eqnarray}
u_i(s^*_i, s_{-i}^*) & \geq & u_i(s_i, s_{-i}^*)
\end{eqnarray} 

In this research, as the players are symmetric and all players have a linear demand function $p = A - bx$ with a constant cost $c$, the Nash Equilibrium can be found as follows: 

\begin{eqnarray}
	\frac{(A-c)^2}{bn_C(m+1)^2} = \frac{D}{n_C(m+1)^2}
\end{eqnarray}

where $D = \frac{(A-c)^2}{b}$, $n_C$ is the number of players in coalition $C$, and $m$ is the number of coalitions in the game. 

\subsection{Integer Partition (IP)}

This research has referenced to the Integer Partition (IP) algorithm developed by Rahwan et al. The algorithm first creates a multiset of positive integer whose sum is $n$. For example, $n=8$, the multiset includes $[4, 3, 1], [2, 2, 2, 2]$ and $[3, 3, 2]$. Then it searches the space by coalition size, and eliminates the search spaces where their maximum utility is smaller than the lower bound. 

Our version of IP will enumerate through all the multisets and compute a certain value for them which the computation will be explained in section \ref{sec:dv-func} and \ref{sec:pv-func}. 

A huge advantage of enumerating throught the multisets is to avoid enumerating the set of coalition structure's n players, whose size is larger than $(\frac{n}{\log n })^n$, while the multisets of n has a much smaller size of at most $e^{\Theta(\sqrt{n})}$. 

In the following sections, we will define two value functions $V^D$ and $V^P$, which will be named as the default value function and the pessimistically value function respectively. These values are responsible for defining the stable CS. 

\subsection{Default value function}
\label{sec:dv-func}

The default value function $V^D(S, i)$ is the reward received by any player in a size-$i$ coalition when the CS is fixed. 
For example, $V^D([3], 3) = D/6}$, $V^D([2, 1], 2) = D/18$, and $V^D([2, 1], 1) = D/9$. 

\subsection{Pessimistically value function}
\label{sec:pv-func}

The pessimistically anticipated value function $V^P(S, i)$ is the pessimisitically anticipated reward (PAR) a player in a size-$i$ coalition might eventually obtain, after any sequence of incentivized betrays by any players.

By incentivized betrayals, we refer to a betrayal initiated by a player such that their PAR with the new CS is strictly better than her default value with the current CS. 

Since coalitions can only be split but not merge in the dynamic betrayal model, $V^P$ can be defined recursively. Let $U$ be the IP $[1, 1, \dots, 1]$, where $1$ occurs $n$ times. This corresponds to the CS where each player forms a coalition by itself. Since no further betrayal can occur from U, $V^P(U, 1)=V^D(U, 1)$. Then $V^P(S, i)$ for any IP $S \neq U$ and $i \in S$ will be defined recursively, going from IPs with the most number of coalitions to the least (the grand coalition).

Before we move forward, we will define a few notation to simplify the presentation. 

\subsubsection{Removal}

Given a multiset $S$ and $i\in S$, let $S - i$ denote the multiset formed by removing S from one occurences of $i$. For example, when $S=[2,2,2,2,1] $, $S-2=[2,2,2,1]$ and $S-1=[2,2,2,2]$. 

\subsubsection{Splitting}

Given a multiset $S$, $i \in S$ and $1 \leq j < i$, let $B(S, i, j)$ 

\subsection{Example}

\subsection{Sequential Blocking}



\subsection{Tables and Figures}

Use the `\texttt{table}' environment (or its variant `\texttt{table*}')
in combination with the `\texttt{tabular}' environment to typeset tables
as floating objects. The `\texttt{aamas}' document class includes the 
`\texttt{booktabs}' package for preparing high-quality tables. Tables 
are often placed at the top of a page near their initial cite, as done 
here for Table~\ref{tab:locations}.

\begin{table}[t]
	\caption{Locations of the first five editions of AAMAS}
	\label{tab:locations}
	\begin{tabular}{rll}\toprule
		\textit{Year} & \textit{City} & \textit{Country} \\ \midrule
		2002 & Bologna & Italy \\
		2003 & Melbourne & Australia \\
		2004 & New York City & USA \\
		2005 & Utrecht & The Netherlands \\
		2006 & Hakodate & Japan \\ \bottomrule
	\end{tabular}
\end{table}

The caption of a table should be placed \emph{above} the table. 
Always use the `\verb|\midrule|' command to separate header rows from 
data rows, and use it only for this purpose. This enables assistive 
technologies to recognise table headers and support their users in 
navigating tables more easily.

%%% The following command should be issued somewhere in the first column 
%%% of the final page of your paper.
\balance

Use the `\texttt{figure}' environment for figures. If your figure 
contains third-party material, make sure to clearly identify it as such.
Every figure should include a caption, and this caption should be placed 
\emph{below} the figure itself, as shown here for Figure~\ref{fig:logo}.

\begin{figure}[h]
  \centering
  \includegraphics[width=0.75\linewidth]{aamas2026logo}
  \caption{The logo of AAMAS 2026}
  \label{fig:logo}
  \Description{Logo of AAMAS 2026 -- The 25th International Conference on Autonomous Agents and Multiagent Systems.}
\end{figure}

In addition, every figure should also have a figure description, unless
it is purely decorative. Use the `\verb|\Description|' command for this 
purpose. These descriptions will not be printed but can be used to 
convey what's in an image to someone who cannot see it. They are also 
used by search engine crawlers for indexing images, and when images 
cannot be loaded. A figure description must consist of unformatted plain 
text of up to 2000~characters. For example, the definition of 
Figure~\ref{fig:logo} in the source file of this document includes the 
following description: ``Logo of AAMAS 2026 -- The 25th International Conference on Autonomous Agents and Multiagent Systems.'' For more information on how best to write figure descriptions 
and why doing so is important, consult the information available here: 
%
\begin{center}
\url{https://www.acm.org/publications/taps/describing-figures/}
\end{center}
%
The use of colour in figures and graphs is permitted, provided they 
remain readable when printed in greyscale and provided they are 
intelligible also for people with a colour vision deficiency.

%%%%%%%%%%%%%%%%%%%%%%%%%%%%%%%%%%%%%%%%%%%%%%%%%%%%%%%%%%%%%%%%%%%%%%%%

\section{Citations and References}
  
The use of the \BibTeX\ to prepare your list of references is highly 
recommended. To include the references at the end of your document, put 
the following two commands just before the `\verb|\end{document}|' 
command in your source file:
%
\begin{verbatim}
   \bibliographystyle{ACM-Reference-Format}
   \bibliography{sample}
\end{verbatim}
%
Here we assume that `\texttt{sample.bib}' is the name of your 
\BibTeX\ file. Use the `\verb|\cite|' command to produce citations 
to your references. Here are a few examples for citations of journal 
articles~\cite{GrKr96,WoJe95}, books~\cite{Knu97}, articles in 
conference proceedings~\cite{Hag1993}, technical reports~\cite{Har78},
Master's and PhD theses~\cite{Ani03,Cla85}, online videos~\cite{Oba08}, 
datasets~\cite{AnMC13}, and patents~\cite{Sci09}. Both citations and 
references are numbered by default. 

Make sure you provide complete and correct bibliographic information 
for all your references, and list authors with their full names 
(``Donald E.\ Knuth'') rather than just initials (``D.\ E.\ Knuth''). 

%%%%%%%%%%%%%%%%%%%%%%%%%%%%%%%%%%%%%%%%%%%%%%%%%%%%%%%%%%%%%%%%%%%%%%%%

%%% The acknowledgments section is defined using the "acks" environment
%%% (rather than an unnumbered section). The use of this environment 
%%% ensures the proper identification of the section in the article 
%%% metadata as well as the consistent spelling of the heading.

\begin{acks}
If you wish to include any acknowledgments in your paper (e.g., to 
people or funding agencies), please do so using the `\texttt{acks}' 
environment. Note that the text of your acknowledgments will be omitted
if you compile your document with the `\texttt{anonymous}' option.
\end{acks}

%%%%%%%%%%%%%%%%%%%%%%%%%%%%%%%%%%%%%%%%%%%%%%%%%%%%%%%%%%%%%%%%%%%%%%%%

%%% The next two lines define, first, the bibliography style to be 
%%% applied, and, second, the bibliography file to be used.

\bibliographystyle{ACM-Reference-Format} 
\bibliography{sample}

%%%%%%%%%%%%%%%%%%%%%%%%%%%%%%%%%%%%%%%%%%%%%%%%%%%%%%%%%%%%%%%%%%%%%%%%

\end{document}

%%%%%%%%%%%%%%%%%%%%%%%%%%%%%%%%%%%%%%%%%%%%%%%%%%%%%%%%%%%%%%%%%%%%%%%%

